 March 2007 %%%%%%%%%%%%%%%%
\documentclass[12pt]{article}
\usepackage{a4wide}

\newcommand{\al}{$<$}
\newcommand{\ar}{$>$}

\parindent 0pt
\parskip 6pt

\begin{document}

\thispagestyle{empty}

\rightline{\large\al\emph{name}\ar}
\medskip
\rightline{\large\al\emph{College}\ar}
\medskip
\rightline{\large\al\emph{CRSID}\ar}

\vfil

\centerline{\large Part II Project Proposal}
\vspace{0.4in}
\centerline{\Large\bf Verified functional data structures and algorithms in Agda}
\vspace{0.3in}
\centerline{\large \al\emph{10th October 2016}\ar}

\vfil

{\bf Project Originator:} Razvan Kusztos

\vspace{0.1in}

{\bf Resources Required:} See attached Project Resource Form

\vspace{0.5in}

{\bf Project Supervisor:} Dr. Timothy Griffin

\vspace{0.2in}

{\bf Signature:}

\vspace{0.5in}

{\bf Director of Studies:}  Chris Hadley

\vspace{0.2in}

{\bf Signature:}

\vspace{0.5in}

{\bf Overseers:} Markus Kuhn and Peter Sewell

\vspace{0.2in}

{\bf Signatures:}

\vfil
\eject

\section*{Introduction and Description of the Work}


Functional data structures are persistent data structures. That is,
whenever a change occurs to the data structure, the programmer should expect to have
both the previous and the new version available. This is different from the usual
imperative approach, where updates are destructive and yield ephemeral data structures.
Although for long it was a consensus that the functional data structures are not
as efficient as their imperative counterparts, Okasaki's book, 'Purely Functional
Data Structures' goes in depth to optimizing structures for this paradigm.
It is coded in a Standard ML and provides informal proofs.

Another approach to providing proofs to such data structures is by using languages,
such as Coq, HOL or Agda. During this project I aim to use Agda as it is the one
to help me learn about depedent types.

Agda is a functional, dependently typed programming language. In this setting,
types can depend on values, allowing for more expressive type declaration.
Agda's environment allows the users the ability to write both the code and the
proofs that have as objects entities in the code in the same language.

Defining data structures and functions in a dependently typed programming language
opens up new possiblities for better expressivity. Invariants can now be part of
the typing system. A great example of this that is already part of the literature
is the implementation or Red-Black trees, which can maintain all five invariants
in the typing declaration. This ensures that any update operation on a Red-Black
tree will continue to maintain these invariants.

Agda was originally developed as a programming language, quite similar in syntax
to Haskell, but kept as simple as possible. The dependently typed environment
allows, via the Curry-Howard correspondence to create human-readable proofs.
Therefore, in the previous example, we can also demonstrate in the same programming
language that the operations that are implemented not only keep the invariants, but
also yield the right results and have the expected mathematical properties.

Although not designed with this goal in mind, if the typing information is
sufficient, Agda can act not only as an interactive prover, but the typing information
can actually help the coding process by using tools such as automatic case split and
expressive goal information.
Moreover, support of Unicode characters in the compiler front-end allows for
even better readability.

The Agda Library is not extensively documented and is a work in progress. It would
be great if via this project I can not only demonstrate the power of this language,
but also contribute with more efficient data structures for the standard library.
Some work include the afore mentioned Red Black Trees and AVL trees (2),


\section*{Resources Required}

I will undertake this project on my laptop + specs

\section*{Starting Point}

At the beginning of this project I have some experience with functional programming
languages (SML from the Foundations of Computer Science course in Part IA) and Scala
(from a Udacity course). Basics of structural induction, which will almost certainly
be a central part are gained from the Semantics of Programming Languages course.
I will have to familirize myself with Agda and most recent advents in terms of
functional data structures, as well as relevant chapters from the
Denotational Semantics and Type Theory courses in Part II.
No prior experience with Agda, dependent types and proofs in this environment.

\section*{Substance and Structure of the Project}

The main structure of the project is gradually building up to more and more
specialized data structures that could be reused in Agda libraries and proving
properties about their correctness and mathematical properties

Okasaki's book describes a mechanism for optimizing functional data structures
that were thought to be a lot inferior to the imperative counter-parts.
A general purpose data structure that stems from okasaki's principles is the
Finger Tree(3). This will be implemented, along with other basic data structures to
illustrate the power of dependent typing in action. Finger Trees will then be used
to implement more powerful data structures, such as Random Access Lists or
Heaps and prove properties of them to be correct. Then, benchmarks will be ran to
show that they are indeed more efficient than what the standard library already provides.
The project will aim to build more complex data structures over general
purpose ones, showing a software engineering approach that ensures correctness
by using dependent types.

Finger Trees have been implemented in Coq (3) although the paper does not describe
any recursive proofs, which are the most important. There is not any implementation
yet that provides size information on the Finger Trees. I hope that, via this
dissertation I can provide a complete dependent implementation of Finger Trees.
Once they are proved correct, I can build up towards any data structure that
their monoidal structure allows. Some examples in the literature are heaps,
sorted sequences, concatenable dequeues etc(3)

The objective of this project is to ultimately write code that is proven to be
bug free and efficient such that it can at some point be made avilable to the
developers of the Agda standard library.
In case the Finger Trees dependent implementation appears to be unfeasible,
there is a fallback plan on starting to implement Okasaki's numerical representation
inspired data structures, defined in Chapter 9 of []. This will have the same
structure as before, building a layering of proved data structures.

The project will start with an introduction section which will provide all the
necessary background for understanding dependently typed programming, the
data structures that are chosen and what work has currently been done in this
field, as well as an introduction to Agda and the libraries that will be used.

The first chapter will be devoted to showing how dependent typing in Agda can be
used to ensure correctness by following through the simple classic example of
vectors, which has been studied before in the past, introducing the concepts that
will be used in the Implementation phase. This will be shown in contrast with
the classic functional list implmentation. Proofs of associatativy of concatenation
and distributivity of length over concatentation will also illustrate how
proofs and implementation can be incorporated in the same environment with Agda.

The implementation phase will introduce Finger Trees and also provide their complete
implementation, as well as all the associated concepts. On top of this,
I will implement a selection of more specialized data structures. It will
be interesting to see how proving properties about the specialized data structures
require recursively proofs about the base data structures. There are no previous
implementations of data structures such as maps or priority queue that can be reused
and also guarantee the amortized cost complexity of the structures I propose.

Evaluation will constist of output of the Agda type checker, showing that the
code is indeed correct, as well as output of a test program. Numerical results
can be derived from the runtime of random tests of increasing lengths, giving
insights into the complexity.
A qualitative discussion of using dependent types for programming in general
can also illustrated with examples from the previous chapters.

\section*{Variants}

Depending on the success of transposing as much of the information as possible
in the type of the data structures, it might turn out the other data structures
are more illustrative to the process, such as perhaps maps and sets, in which
case those will be the chosen data structures to evolve Finger Trees to.
As already mentioned, if the complete dependent implementation of Finger Trees
becomse too cumbersome, the Okasaki's numerical representation inspired
data structures are a good and similar in spirit alternative. I would like to
stick to the Finger Trees as much as possible because they are already established
as a swiss army knife of functional data structures
Some work could be done on trying to specialize on of these structures as much
as possible to see how much expressivity can be derived in the chosen paradigm.
One example of this is found in [] exerices, where a sublist is defined by a
type that contains information about the underlying list. This could be extended
to sets

\section*{Success Criterion}

For the project to be deemed a success the following items must be
successfully completed.

\begin{enumerate}

\item Agda and Dependent typing should have a convincing justification.

\item The basic, general-purpose data structure should be implemented, illustrating
the advantages de dependent typing brings, as well as an explanation of why they
are more efficient than standard approaches.

\item This data structure will be supported by proofs.

\item More complex data structures can be built on top. This will enforce
correctness throughout.

\item Providing proofs of various mathematical properties of the data structure,
whether they exhibit any algebraic structure properties, whether invariants hold,
such as insertion of an element does indeed insert the element into the data structure.

\item Further proofs of the invariants that are held by the typing system, proving
that there hasn't been made any mistake while writing the type annotations.

\item The data structures should be packaged in a reusable library that should be
made available online.

\item Considerations of complexity should be provided.

\item The dissertation must be planned and written.

\end{enumerate}

\section*{Timetable and Milestones}

\subsection*{Weeks 1 to 5}

These first weeks will be used for learning and reading time mostly. I aim to
start writing the introduction material that will cover agda and dependent typing
introductions, marking my progress as I familiarize myself with the language.
By the end of this period, I plan to find out whether a dependently typed implementation
of Finger Trees is indeed feasible.

\subsection*{Weeks 6 and 7}

Meeting with the supervisor to show the progress, and also set the exact workflow
through the data structures

\subsection*{Weeks 8 to 10}

Finish implementation of the base data structures and proofs + write-up in the
dissertation

\subsection*{Weeks 11 and 12}

Finish implementation of the complex data structures and proofs.
\subsection*{Weeks 13 to 19 (including Easter vacation)}

Write-up in the dissertation or the previous work. Try to find in literature and
example where I can specialize the data structure so much that would truly show
the power of the chosen setup (if time). Everything done in this period should
be writen down.

\subsection*{Weeks 20 to 26}

Evaluation phase, carefull discussion with examples from literature,
Run-time evaluation
Type-checking agda output.
Interacitve example of coding in a dependently typed programming language.

\subsection*{Weeks 27 to 31}

Brush up, write bibliography, add any information that was missed, add information
that was discovered after the writing up and has not been included yet.
Restructure to make the dissertation more readable

\subsection*{Weeks 32 to 33}

Further improvements related to the material. Meeting with supervisor.

\subsection*{Week 34}

Milestone: Submission of Dissertation.

\section{References}

\end
\end{document}
