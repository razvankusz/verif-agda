\documentclass[11pt]{article}

\title{Personal Statement}
\author{Christopher Lesniewski-Laas}
\date{January 2001}

\pagestyle{myheadings}
\markright{Christopher Lesniewski-Laas \hfil Personal Statement}


\begin{document}
\maketitle



The pursuit of learning and knowledge has always been central to
myself and my family, and I'd like to continue my family's tradition
by contributing to the world in the intellectual domain.  My father,
who immigrated from Poland to America at the age of fifteen, toiled
long to raise himself from nothing to a respected lawyer and diplomat;
I've always felt that I should respect the head start I was given by
moving ever forward in the arts and sciences.

As a student in grade school and high school, I always felt a drive to
learn and understand as much as possible.  My family never pushed me
to excel, but supported my avid reading and study; although, as
lawyers both, they weren't familiar with the mathematics and physics
that I preferred, they did their best to help me grow in whichever
direction I chose.  I discovered computer programming, and,
discovering that I had some natural talent for it, I found a job
developing software for a local ISP.

By my junior year in high school, I had exhausted my high school's
curriculum I applied to one university, MIT, and was accepted early
admission.  MIT is a wonder.  Here, I've always been able to learn as
quickly as I wish in my chosen fields --- mathematics, physics, and
electrical engineering and computer science (EECS) --- without being
hampered by rigid prerequisites.  When I wished to study Einstein's
General Relativity (class number 8.962), I simply took the class, and
filled in the holes in my knowledge as I went along.  MIT is populated
by a host of colorful characters, and I've found no end of tempting
distractions from my studies: in my free time, I regularly go rock
climbing, work on robotics, and skydive.

Lately, most of my time is spent on my various service and research
projects, however.  I am an officer of the local chapter of Eta Kappa
Nu, the national EECS honor society, and I serve as the
editor-in-chief of the Underground Guide to Course 6, which is the
student-run course evaluation guide for EECS classes at MIT.  As a
member of the well-known Student Information Processing Board, a
service group devoted to improving students' access to computers at
MIT, I volunteer many hours a week to answer walk-in and phone
questions, to work on tools which benefit the MIT community, and to
teach short classes about topics in practical computer use.  I'm also
given the opportunity to teach in my job as a lab assistant for 6.001,
MIT famed introduction to computer science, where I aid students in
debugging their first programs.

In the professional realm, I've worked both within academia and in
industry: for example, as a sophomore I researched massive
``amorphous'' computing arrays in MIT's Artificial Intelligence Lab,
and as a junior I developed robotic software and hardware for the
haptics company SensAble Technologies, Inc.  For the past year, I have
been occupied with a project of my own: I proposed to replace MIT's
key-card system with a cryptographically secure and privacy-protecting
system, and the I-Campus initiative awarded the project a \$60,000
grant.  As part of my efforts in the project, I accepted a job in the
cryptography group of Microsoft Research, developing the protocols
which would be used in the system.  The running of this project has
been often frustrating, but immensely rewarding nevertheless.

For the long term, my fundamental goals are the same as ever: continue
to learn about the universe.  Where the existing knowledge peters out,
I'll research.  Because several topics in EECS have become my realm of
expertise, I hope to do research at the Lab for Computer Science at
MIT, and work towards a Ph.D.; however, hard times for my family (in
addition to two siblings coming up to college age) mean that I can't
expect any more financial assistance from them, and have to stand on
my own from now on.  I hope to focus on my research while in graduate
school: several fields of computer science especially hold my
interest.  Cryptography, the art of implementing trust models, brings
together human relations (the trust contracts that people hold with
each other), abstract mathematics, and computer programming closer
than ever before.  Computer architectures, the art of designing
circuits which perform high-speed computation, have enabled the entire
current Internet boom.  And the nascent field of distributed systems,
the art of creating computer systems which span the network, will
propel the Internet into its next stage of evolution.  Most such
systems are totally inefficient as they are, and only hold together
(when they do) because demand is light.  However, as this century
brings the network to ever greater heights, there will be a need for
scalable and secure systems, and my research will help to fill that
need.  To be able to make a contribution in such an exciting area
would please me to no end, and I am thankful to have such a wonderful
opportunity.





\end{document}
