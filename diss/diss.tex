% The master copy of this demo dissertation is held on my filespace
% on the cl file serve (/homes/mr/teaching/demodissert/)

% Last updated by MR on 2 August 2001

\documentclass[12pt,twoside,notitlepage]{report}

\usepackage{a4}
\usepackage{verbatim}
\usepackage{fontspec}
\setmainfont{DejaVu Sans}
\setmonofont{DejaVu Sans Mono}
\input{epsf}                            % to allow postscript inclusions
% On thor and CUS read top of file:
%     /opt/TeX/lib/texmf/tex/dvips/epsf.sty
% On CL machines read:
%     /usr/lib/tex/macros/dvips/epsf.tex



\raggedbottom                           % try to avoid widows and orphans
\sloppy
\clubpenalty1000%
\widowpenalty1000%

\addtolength{\oddsidemargin}{6mm}       % adjust margins
\addtolength{\evensidemargin}{-8mm}

\renewcommand{\baselinestretch}{1.1}    % adjust line spacing to make
                                        % more readable

\begin{document}

\bibliographystyle{plain}


%%%%%%%%%%%%%%%%%%%%%%%%%%%%%%%%%%%%%%%%%%%%%%%%%%%%%%%%%%%%%%%%%%%%%%%%
% Title


\pagestyle{empty}

\hfill{\LARGE \bf Razvan Kusztos}

\vspace*{60mm}
\begin{center}
\Huge
{\bf Verified functional datastructures in Agda} \\
\vspace*{5mm}
Diploma in Computer Science \\
\vspace*{5mm}
Girton College \\
\vspace*{5mm}
\today  % today's date
\end{center}

\cleardoublepage

%%%%%%%%%%%%%%%%%%%%%%%%%%%%%%%%%%%%%%%%%%%%%%%%%%%%%%%%%%%%%%%%%%%%%%%%%%%%%%
% Proforma, table of contents and list of figures

\setcounter{page}{1}
\pagenumbering{roman}
\pagestyle{plain}

\chapter*{Proforma}

{\large
\begin{tabular}{ll}
Name:               & \bf Razvan Kusztos                       \\
College:            & \bf Girton College                     \\
Project Title:      & \bf Verified functional datasturcures and algorithm in Agda \\
Examination:        & \bf Part II Project        \\
Word Count:         & \bf 0\footnotemark[1]
(well less than the 12000 limit) \\
Project Originator: & Dr Timothy Griffin                    \\
Supervisor:         & Dr Timothy Griffin                    \\
\end{tabular}
}
\footnotetext[1]{This word count was computed
by {\tt detex diss.tex | tr -cd '0-9A-Za-z $\tt\backslash$n' | wc -w}
}
\stepcounter{footnote}

\newpage
\section*{Declaration}

I, [Name] of [College], being a candidate for Part II of the Computer
Science Tripos [or the Diploma in Computer Science], hereby declare
that this dissertation and the work described in it are my own work,
unaided except as may be specified below, and that the dissertation
does not contain material that has already been used to any substantial
extent for a comparable purpose.

\bigskip
\leftline{Signed [signature]}

\medskip
\leftline{Date [date]}

\cleardoublepage

\tableofcontents

\listoffigures

\newpage
\section*{Acknowledgements}

This document owes much to an earlier version written by Simon Moore
\cite{moore95}.  His help, encouragement and advice was greatly
appreciated.

%%%%%%%%%%%%%%%%%%%%%%%%%%%%%%%%%%%%%%%%%%%%%%%%%%%%%%%%%%%%%%%%%%%%%%%
% now for the chapters

\cleardoublepage        % just to make sure before the page numbering
                        % is changed

\setcounter{page}{1}
\pagenumbering{arabic}
\pagestyle{headings}

\chapter{Introduction}

\section{Functional Datastructures}

There has always been an imballance between the use of functional programming versus imperative programming both in
industry, as well as in terms of available resource. Functional programming has often been ruled out in the
past because of it running slow, or simply because it was stigmatised as belonging into academia, regardless of its
properties \cite{landin}. However, this paradigm
is being introduced now at the forefront of business development. This is because of the persistency\footnotemark[0] of data-structures,
useful for the ever-present multicore environment. The reliablity aspect, formal verification and
keeping runtime errors to a minimum have also been relevant.
The issue of the runtime speed has been addressed gracefully by Okasaki \cite{oka}, which, introduced a reusable concept
that can help designers build efficient data structures: the implicit recursive slowdown.

\section{Dependent Typing}

An important breakthrough in writing verifiably correct code is the introduction of dependent types.\cite{achipala}
In this setting the distinction between types and values becomes blurry, allowing us to define types that depend on values.


A good practical motivation is performing the sum of two vectors. The usual programming paradigm would be (in pseudocode): \\

\begin{verbatim}
def sum (l1, l2):
  if (l1.length != l2.length)
    raise ListsNotEqualException;
  ...
\end{verbatim}

This can cause a runtime error and arguably disrupts the logical flow of the program. In a program that
supports dependent types, we can construct lists that are both parametrized by a variable (as in the usual
polymorphic programming), but also 'indexed'.

The usual definiton of lists would be (in Agda - but easilty and functional language)

\begin{verbatim}
  data List (A : Set) : Set where
    nil : List A
    _∷_ : A → List A → List A
 \end{verbatim}

Compare this with the dependent definition:

\begin{verbatim}
  data Vec (A : Set) : ℕ → Set where
    nil : Vec A 0
    _∷_ : ∀ {n : ℕ} → A → Vec A n → Vec A (n + 1)
\end{verbatim}

This allows us to write functions that require a 'proof' that the two vectors are equal.

\begin{verbatim}
  sum : ∀ {n : ℕ} → Vec ℕ n → Vec ℕ n → Vec ℕ n
  sum xs ys = ?
\end{verbatim}
If this is not obvious from the context, the program will not type check.
The developer is forced to only write correct programs.

Further details about agda syntax will be provided in section (Introduction to Agda)

\section{Nested Types}

Another way of maintaing invariants throughout the program is the trick or 'irregular'
or 'nested' datatypes. They allow forcing strong structural invariants on the datastructure
and have gained interest because of their practical implications. \cite{birdmeertens}
Some difficulties come up in recursive calls or inductive proofs, fact which will be
covered and discussed in section (TODO implementation/section_nest_example).

The motivation for introducing this concepts is that it allows maintaining
the invariant of a full tree where information is stored only at the leafs.

\begin{verbatim}
  data BinTree (A : Set) : Set where
    empty : BinTree A
    single : A → BinTree A
    deep :  BinTree (Node A) → BinTree A
\end{verbatim}
Where Node is simply:
\begin{verbatim}
  data Node (A : Set) : Set where
    node : A → A → Node A
\end{verbatim}

It can be seen from the declaration that the structure will be forced to be a sequence of deep
constructors, followed by either a single (Nodeⁿ A) or an empty constructor. The number
of elements is forced therefore to be a power of two (2ⁿ) making it equivalent with the leafs of a full
binary tree.

This dissertation is concerned with 2-3 trees, the basis for the FingerTrees, which
will be studied in detail in the Implementation section. They are an example to show arising problems
when proving properties of nested and dependent typed structures, what limits are imposed and how
some of them can be overcome.

\section{Introduction to Agda}

Agda is a dependently typed programming language, developed in the spirit of Haskell, kept as
simple as possible \cite{ulf}.
All the previous examples are written in Agda, and their syntax and the newly introduced syntax
will be described as we go on.
Along other programming languages like Coq \cite{coq} or Isabelle \cite{isabelle}, Agda is used as
an interactive (or automatic) theorem prover. What makes it different from the two previously
mentioned system is the ability to write the code and the proofs in the same environment. Its relative
simplicity also motivated its use in this project.


\section{Overview of the files}

\section{Building the document}

\subsection{The makefile}

To simplify the calls to {\tt latex} and {\tt bibtex},
a makefile has been provided, see Appendix %%~\ref{makefile}.
It provides the following facilities:

\begin{itemize}

\item{\tt make} \\
 Display help information.


\end{itemize}




\cleardoublepage



\chapter{Preparation}

This chapter is empty!


\cleardoublepage
\chapter{Implementation}

\section{Verbatim text}

Verbatim text can be included using \verb|\begin{verbatim}| and
\verb|\end{verbatim}|. I normally use a slightly smaller font and
often squeeze the lines a little closer together, as in:

{\renewcommand{\baselinestretch}{0.8}\small\begin{verbatim}
GET "libhdr"

GLOBAL { count:200; all  }

LET try(ld, row, rd) BE TEST row=all
                        THEN count := count + 1
                        ELSE { LET poss = all & ~(ld | row | rd)
                               UNTIL poss=0 DO
                               { LET p = poss & -poss
                                 poss := poss - p
                                 try(ld+p << 1, row+p, rd+p >> 1)
                               }
                             }
LET start() = VALOF
{ all := 1
  FOR i = 1 TO 12 DO
  { count := 0
    try(0, 0, 0)
    writef("Number of solutions to %i2-queens is %i5*n", i, count)
    all := 2*all + 1
  }
  RESULTIS 0
}
\end{verbatim}
}

\section{Tables}

\begin{samepage}
Here is a simple example\footnote{A footnote} of a table.

\begin{center}
\begin{tabular}{l|c|r}
Left      & Centred & Right \\
Justified &         & Justified \\[3mm]
%\hline\\%[-2mm]
First     & A       & XXX \\
Second    & AA      & XX  \\
Last      & AAA     & X   \\
\end{tabular}
\end{center}

\noindent
There is another example table in the proforma.
\end{samepage}

\section{Simple diagrams}




\cleardoublepage

\chapter{Evaluation}

\section{Printing and binding}

If you have access to a laser printer that can print on two sides, you
can use it to print two copies of your dissertation and then get them
bound by the Computer Laboratory Bookshop. Otherwise, print your
dissertation single sided and get the Bookshop to copy and bind it double
sided.


Better printing quality can sometimes be obtained by giving the
Bookshop an MSDOS 1.44~Mbyte 3.5" floppy disc containing the
Postscript form of your dissertation. If the file is too large a
compressed version with {\tt zip} but not {\tt gnuzip} nor {\tt
compress} is acceptable. However they prefer the uncompressed form if
possible. From my experience I do not recommend this method.

\subsection{Things to note}

\begin{itemize}
\item Ensure that there are the correct number of blank pages inserted
so that each double sided page has a front and a back.  So, for
example, the title page must be followed by an absolutely blank page
(not even a page number).

\item Submitted postscript introduces more potential problems.
Therefore you must either allow two iterations of the binding process
(once in a digital form, falling back to a second, paper, submission if
necessary) or submit both paper and electronic versions.

\item There may be unexpected problems with fonts.

\end{itemize}

\section{Further information}

See the Computer Lab's world wide web pages at URL:

{\tt http://www.cl.cam.ac.uk/TeXdoc/TeXdocs.html}


\cleardoublepage
\chapter{Conclusion}

I hope that this rough guide to writing a dissertation is \LaTeX\ has
been helpful and saved you time.


\end{document}
%
% \cleardoublepage
%
% %%%%%%%%%%%%%%%%%%%%%%%%%%%%%%%%%%%%%%%%%%%%%%%%%%%%%%%%%%%%%%%%%%%%%
% % the bibliography
%
% \addcontentsline{toc}{chapter}{Bibliography}
% \bibliography{refs}
% \cleardoublepage
%
% %%%%%%%%%%%%%%%%%%%%%%%%%%%%%%%%%%%%%%%%%%%%%%%%%%%%%%%%%%%%%%%%%%%%%
% % the appendices
% \appendix
%
% \chapter{Latex source}
%
% \section{diss.tex}
% {\scriptsize\verbatiminput{diss.tex}}
%
% \section{proposal.tex}
% {\scriptsize\verbatiminput{proposal.tex}}
%
% \section{propbody.tex}
% {\scriptsize\verbatiminput{propbody.tex}}
%
%
%
% \cleardoublepage
%
% \chapter{Makefile}
%
% \section{\label{makefile}Makefile}
% {\scriptsize\verbatiminput{makefile.txt}}
%
% \section{refs.bib}
% {\scriptsize\verbatiminput{refs.bib}}
%
%
% \cleardoublepage
%
% \chapter{Project Proposal}
%
% \input{propbody}
%
% \end{document}
