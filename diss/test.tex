%% ODER: format ==         = "\mathrel{==}"
%% ODER: format /=         = "\neq "
%
%
\makeatletter
\@ifundefined{lhs2tex.lhs2tex.sty.read}%
  {\@namedef{lhs2tex.lhs2tex.sty.read}{}%
   \newcommand\SkipToFmtEnd{}%
   \newcommand\EndFmtInput{}%
   \long\def\SkipToFmtEnd#1\EndFmtInput{}%
  }\SkipToFmtEnd

\newcommand\ReadOnlyOnce[1]{\@ifundefined{#1}{\@namedef{#1}{}}\SkipToFmtEnd}
\usepackage{amstext}
\usepackage{amssymb}
\usepackage{stmaryrd}
\DeclareFontFamily{OT1}{cmtex}{}
\DeclareFontShape{OT1}{cmtex}{m}{n}
  {<5><6><7><8>cmtex8
   <9>cmtex9
   <10><10.95><12><14.4><17.28><20.74><24.88>cmtex10}{}
\DeclareFontShape{OT1}{cmtex}{m}{it}
  {<-> ssub * cmtt/m/it}{}
\newcommand{\texfamily}{\fontfamily{cmtex}\selectfont}
\DeclareFontShape{OT1}{cmtt}{bx}{n}
  {<5><6><7><8>cmtt8
   <9>cmbtt9
   <10><10.95><12><14.4><17.28><20.74><24.88>cmbtt10}{}
\DeclareFontShape{OT1}{cmtex}{bx}{n}
  {<-> ssub * cmtt/bx/n}{}
\newcommand{\tex}[1]{\text{\texfamily#1}}	% NEU

\newcommand{\Sp}{\hskip.33334em\relax}


\newcommand{\Conid}[1]{\mathit{#1}}
\newcommand{\Varid}[1]{\mathit{#1}}
\newcommand{\anonymous}{\kern0.06em \vbox{\hrule\@width.5em}}
\newcommand{\plus}{\mathbin{+\!\!\!+}}
\newcommand{\bind}{\mathbin{>\!\!\!>\mkern-6.7mu=}}
\newcommand{\rbind}{\mathbin{=\mkern-6.7mu<\!\!\!<}}% suggested by Neil Mitchell
\newcommand{\sequ}{\mathbin{>\!\!\!>}}
\renewcommand{\leq}{\leqslant}
\renewcommand{\geq}{\geqslant}
\usepackage{polytable}

%mathindent has to be defined
\@ifundefined{mathindent}%
  {\newdimen\mathindent\mathindent\leftmargini}%
  {}%

\def\resethooks{%
  \global\let\SaveRestoreHook\empty
  \global\let\ColumnHook\empty}
\newcommand*{\savecolumns}[1][default]%
  {\g@addto@macro\SaveRestoreHook{\savecolumns[#1]}}
\newcommand*{\restorecolumns}[1][default]%
  {\g@addto@macro\SaveRestoreHook{\restorecolumns[#1]}}
\newcommand*{\aligncolumn}[2]%
  {\g@addto@macro\ColumnHook{\column{#1}{#2}}}

\resethooks

\newcommand{\onelinecommentchars}{\quad-{}- }
\newcommand{\commentbeginchars}{\enskip\{-}
\newcommand{\commentendchars}{-\}\enskip}

\newcommand{\visiblecomments}{%
  \let\onelinecomment=\onelinecommentchars
  \let\commentbegin=\commentbeginchars
  \let\commentend=\commentendchars}

\newcommand{\invisiblecomments}{%
  \let\onelinecomment=\empty
  \let\commentbegin=\empty
  \let\commentend=\empty}

\visiblecomments

\newlength{\blanklineskip}
\setlength{\blanklineskip}{0.66084ex}

\newcommand{\hsindent}[1]{\quad}% default is fixed indentation
\let\hspre\empty
\let\hspost\empty
\newcommand{\NB}{\textbf{NB}}
\newcommand{\Todo}[1]{$\langle$\textbf{To do:}~#1$\rangle$}

\EndFmtInput
\makeatother
%
%
%
%
%
%
% This package provides two environments suitable to take the place
% of hscode, called "plainhscode" and "arrayhscode". 
%
% The plain environment surrounds each code block by vertical space,
% and it uses \abovedisplayskip and \belowdisplayskip to get spacing
% similar to formulas. Note that if these dimensions are changed,
% the spacing around displayed math formulas changes as well.
% All code is indented using \leftskip.
%
% Changed 19.08.2004 to reflect changes in colorcode. Should work with
% CodeGroup.sty.
%
\ReadOnlyOnce{polycode.fmt}%
\makeatletter

\newcommand{\hsnewpar}[1]%
  {{\parskip=0pt\parindent=0pt\par\vskip #1\noindent}}

% can be used, for instance, to redefine the code size, by setting the
% command to \small or something alike
\newcommand{\hscodestyle}{}

% The command \sethscode can be used to switch the code formatting
% behaviour by mapping the hscode environment in the subst directive
% to a new LaTeX environment.

\newcommand{\sethscode}[1]%
  {\expandafter\let\expandafter\hscode\csname #1\endcsname
   \expandafter\let\expandafter\endhscode\csname end#1\endcsname}

% "compatibility" mode restores the non-polycode.fmt layout.

\newenvironment{compathscode}%
  {\par\noindent
   \advance\leftskip\mathindent
   \hscodestyle
   \let\\=\@normalcr
   \let\hspre\(\let\hspost\)%
   \pboxed}%
  {\endpboxed\)%
   \par\noindent
   \ignorespacesafterend}

\newcommand{\compaths}{\sethscode{compathscode}}

% "plain" mode is the proposed default.
% It should now work with \centering.
% This required some changes. The old version
% is still available for reference as oldplainhscode.

\newenvironment{plainhscode}%
  {\hsnewpar\abovedisplayskip
   \advance\leftskip\mathindent
   \hscodestyle
   \let\hspre\(\let\hspost\)%
   \pboxed}%
  {\endpboxed%
   \hsnewpar\belowdisplayskip
   \ignorespacesafterend}

\newenvironment{oldplainhscode}%
  {\hsnewpar\abovedisplayskip
   \advance\leftskip\mathindent
   \hscodestyle
   \let\\=\@normalcr
   \(\pboxed}%
  {\endpboxed\)%
   \hsnewpar\belowdisplayskip
   \ignorespacesafterend}

% Here, we make plainhscode the default environment.

\newcommand{\plainhs}{\sethscode{plainhscode}}
\newcommand{\oldplainhs}{\sethscode{oldplainhscode}}
\plainhs

% The arrayhscode is like plain, but makes use of polytable's
% parray environment which disallows page breaks in code blocks.

\newenvironment{arrayhscode}%
  {\hsnewpar\abovedisplayskip
   \advance\leftskip\mathindent
   \hscodestyle
   \let\\=\@normalcr
   \(\parray}%
  {\endparray\)%
   \hsnewpar\belowdisplayskip
   \ignorespacesafterend}

\newcommand{\arrayhs}{\sethscode{arrayhscode}}

% The mathhscode environment also makes use of polytable's parray 
% environment. It is supposed to be used only inside math mode 
% (I used it to typeset the type rules in my thesis).

\newenvironment{mathhscode}%
  {\parray}{\endparray}

\newcommand{\mathhs}{\sethscode{mathhscode}}

% texths is similar to mathhs, but works in text mode.

\newenvironment{texthscode}%
  {\(\parray}{\endparray\)}

\newcommand{\texths}{\sethscode{texthscode}}

% The framed environment places code in a framed box.

\def\codeframewidth{\arrayrulewidth}
\RequirePackage{calc}

\newenvironment{framedhscode}%
  {\parskip=\abovedisplayskip\par\noindent
   \hscodestyle
   \arrayrulewidth=\codeframewidth
   \tabular{@{}|p{\linewidth-2\arraycolsep-2\arrayrulewidth-2pt}|@{}}%
   \hline\framedhslinecorrect\\{-1.5ex}%
   \let\endoflinesave=\\
   \let\\=\@normalcr
   \(\pboxed}%
  {\endpboxed\)%
   \framedhslinecorrect\endoflinesave{.5ex}\hline
   \endtabular
   \parskip=\belowdisplayskip\par\noindent
   \ignorespacesafterend}

\newcommand{\framedhslinecorrect}[2]%
  {#1[#2]}

\newcommand{\framedhs}{\sethscode{framedhscode}}

% The inlinehscode environment is an experimental environment
% that can be used to typeset displayed code inline.

\newenvironment{inlinehscode}%
  {\(\def\column##1##2{}%
   \let\>\undefined\let\<\undefined\let\\\undefined
   \newcommand\>[1][]{}\newcommand\<[1][]{}\newcommand\\[1][]{}%
   \def\fromto##1##2##3{##3}%
   \def\nextline{}}{\) }%

\newcommand{\inlinehs}{\sethscode{inlinehscode}}

% The joincode environment is a separate environment that
% can be used to surround and thereby connect multiple code
% blocks.

\newenvironment{joincode}%
  {\let\orighscode=\hscode
   \let\origendhscode=\endhscode
   \def\endhscode{\def\hscode{\endgroup\def\@currenvir{hscode}\\}\begingroup}
   %\let\SaveRestoreHook=\empty
   %\let\ColumnHook=\empty
   %\let\resethooks=\empty
   \orighscode\def\hscode{\endgroup\def\@currenvir{hscode}}}%
  {\origendhscode
   \global\let\hscode=\orighscode
   \global\let\endhscode=\origendhscode}%

\makeatother
\EndFmtInput
%

\begin{hscode}\SaveRestoreHook
\column{B}{@{}>{\hspre}l<{\hspost}@{}}%
\column{E}{@{}>{\hspre}l<{\hspost}@{}}%
\>[B]{}\mathbf{module}\;\Varid{test}\;\mathbf{where}{}\<[E]%
\ColumnHook
\end{hscode}\resethooks


\begin{hscode}\SaveRestoreHook
\column{B}{@{}>{\hspre}l<{\hspost}@{}}%
\column{3}{@{}>{\hspre}l<{\hspost}@{}}%
\column{5}{@{}>{\hspre}l<{\hspost}@{}}%
\column{E}{@{}>{\hspre}l<{\hspost}@{}}%
\>[3]{}\Varid{open}\;\mathbf{import}\;\Conid{\Conid{Data}.Nat}{}\<[E]%
\\
\>[3]{}\mathbf{infix}\;\mathrm{4}\;\anonymous \mathbin{==\char95 }{}\<[E]%
\\
\>[3]{}\mathbf{data}\;\anonymous \mathbin{==\char95 }\mathbin{:}\Conid{ℕ}\mathbin{→}\Conid{ℕ}\mathbin{→}\Conid{Set}\;\mathbf{where}{}\<[E]%
\\
\>[3]{}\hsindent{2}{}\<[5]%
\>[5]{}\Varid{ze}\mathbin{:}\Varid{zero}\equiv \Varid{zero}{}\<[E]%
\\
\>[3]{}\hsindent{2}{}\<[5]%
\>[5]{}\Varid{sc}\mathbin{:}\mathbin{∀}\{\mskip1.5mu \Varid{n}\;\Varid{m}\mskip1.5mu\}\mathbin{→}\Varid{n}\equiv \Varid{m}\mathbin{→}(\Varid{suc}\;\Varid{n})\equiv (\Varid{suc}\;\Varid{m}){}\<[E]%
\ColumnHook
\end{hscode}\resethooks

  \begin{hscode}\SaveRestoreHook
\column{B}{@{}>{\hspre}l<{\hspost}@{}}%
\column{3}{@{}>{\hspre}l<{\hspost}@{}}%
\column{E}{@{}>{\hspre}l<{\hspost}@{}}%
\>[3]{}\mathrm{0}\mathbin{-}\Varid{left}\mathbin{:}\mathbin{∀}(\Varid{n}\mathbin{:}\Conid{ℕ})\mathbin{→}(\Varid{zero}\mathbin{+}\Varid{n}\equiv \Varid{n}){}\<[E]%
\\
\>[3]{}\mathrm{0}\mathbin{-}\Varid{left}\;\Varid{zero}\mathrel{=}\Varid{ze}{}\<[E]%
\\
\>[3]{}\mathrm{0}\mathbin{-}\Varid{left}\;(\Varid{suc}\;\Varid{n})\mathrel{=}\Varid{sc}\;(\mathrm{0}\mathbin{-}\Varid{left}\;\Varid{n}){}\<[E]%
\ColumnHook
\end{hscode}\resethooks

\begin{hscode}\SaveRestoreHook
\column{B}{@{}>{\hspre}l<{\hspost}@{}}%
\column{3}{@{}>{\hspre}l<{\hspost}@{}}%
\column{E}{@{}>{\hspre}l<{\hspost}@{}}%
\>[3]{}\Varid{leq}\mathbin{-}\Varid{antisym}\mathbin{:}\mathbin{∀}(\Varid{n}\mathbin{:}\Conid{ℕ})\;(\Varid{m}\mathbin{:}\Conid{ℕ})\mathbin{→}(\Varid{n}\mathbin{≤}\Varid{m})\mathbin{→}(\Varid{m}\mathbin{≤}\Varid{n})\mathbin{→}(\Varid{n}\equiv \Varid{m}){}\<[E]%
\\
\>[3]{}\Varid{leq}\mathbin{-}\Varid{antisym}\;\Varid{zero}\;\Varid{zero}\;\Varid{z}\mathbin{≤}\Varid{n}\;\Varid{z}\mathbin{≤}\Varid{n}\mathrel{=}\Varid{ze}{}\<[E]%
\\
\>[3]{}\Varid{leq}\mathbin{-}\Varid{antisym}\;\Varid{zero}\;(\Varid{suc}\;\Varid{m})\;\Varid{z}\mathbin{≤}\Varid{n}\;(){}\<[E]%
\\
\>[3]{}\Varid{leq}\mathbin{-}\Varid{antisym}\;(\Varid{suc}\;\Varid{n})\;(\Varid{suc}\;\Varid{m})\;(\Varid{s}\mathbin{≤}\Varid{s}\;\Varid{p1})\;(\Varid{s}\mathbin{≤}\Varid{s}\;\Varid{p2})\mathrel{=}\Varid{sc}\;(\Varid{leq}\mathbin{-}\Varid{antisym}\;\Varid{n}\;\Varid{m}\;\Varid{p1}\;\Varid{p2}){}\<[E]%
\ColumnHook
\end{hscode}\resethooks

\begin{hscode}\SaveRestoreHook
\column{B}{@{}>{\hspre}l<{\hspost}@{}}%
\column{3}{@{}>{\hspre}l<{\hspost}@{}}%
\column{E}{@{}>{\hspre}l<{\hspost}@{}}%
\>[3]{}\Varid{open}\;\mathbf{import}\;\Conid{\Conid{Relation}.\Conid{Binary}.PropositionalEquality}{}\<[E]%
\\[\blanklineskip]%
\>[3]{}\mathbin{+}\Varid{assoc}\mathbin{:}\mathbin{∀}(\Varid{x}\;\Varid{y}\;\Varid{z}\mathbin{:}\Conid{ℕ})\mathbin{→}(\Varid{x}\mathbin{+}(\Varid{y}\mathbin{+}\Varid{z}))\mathbin{≡}((\Varid{x}\mathbin{+}\Varid{y})\mathbin{+}\Varid{z}){}\<[E]%
\\
\>[3]{}\mathbin{+}\Varid{assoc}\;\Varid{zero}\;\Varid{y}\;\Varid{z}\mathrel{=}\Varid{refl}{}\<[E]%
\\
\>[3]{}\mathbin{+}\Varid{assoc}\;(\Varid{suc}\;\Varid{x})\;\Varid{y}\;\Varid{z}\;\Varid{rewrite}\mathbin{+}\Varid{assoc}\;\Varid{x}\;\Varid{y}\;\Varid{z}\mathrel{=}\Varid{refl}{}\<[E]%
\ColumnHook
\end{hscode}\resethooks

\begin{hscode}\SaveRestoreHook
\column{B}{@{}>{\hspre}l<{\hspost}@{}}%
\column{3}{@{}>{\hspre}l<{\hspost}@{}}%
\column{E}{@{}>{\hspre}l<{\hspost}@{}}%
\>[3]{}\Varid{open}\;\mathbf{import}\;\Conid{\Conid{Data}.Nat}{}\<[E]%
\ColumnHook
\end{hscode}\resethooks

  This imports the standard library version of natural numbers, which is declared in the exactly same way.
  It allows us to use actual digits for their representations, which can enhance readability.

  Further down are some further proofs we need in order to show natural numbers are commutative.

\begin{hscode}\SaveRestoreHook
\column{B}{@{}>{\hspre}l<{\hspost}@{}}%
\column{3}{@{}>{\hspre}l<{\hspost}@{}}%
\column{E}{@{}>{\hspre}l<{\hspost}@{}}%
\>[3]{}\mathbin{+}\mathrm{0}\mathbin{:}(\Varid{m}\mathbin{:}\Conid{ℕ})\mathbin{→}(\Varid{m}\mathbin{+}\mathrm{0})\mathbin{≡}\Varid{m}{}\<[E]%
\\
\>[3]{}\mathbin{+}\mathrm{0}\;\Varid{zero}\mathrel{=}\Varid{refl}{}\<[E]%
\\
\>[3]{}\mathbin{+}\mathrm{0}\;(\Varid{suc}\;\Varid{m})\;\Varid{rewrite}\mathbin{+}\mathrm{0}\;\Varid{m}\mathrel{=}\Varid{refl}{}\<[E]%
\\[\blanklineskip]%
\>[3]{}\mathbin{+}\Varid{suc}\mathbin{:}\mathbin{∀}(\Varid{x}\;\Varid{y}\mathbin{:}\Conid{ℕ})\mathbin{→}(\Varid{x}\mathbin{+}(\Varid{suc}\;\Varid{y}))\mathbin{≡}(\Varid{suc}\;(\Varid{x}\mathbin{+}\Varid{y})){}\<[E]%
\\
\>[3]{}\mathbin{+}\Varid{suc}\;\Varid{zero}\;\Varid{y}\mathrel{=}\Varid{refl}{}\<[E]%
\\
\>[3]{}\mathbin{+}\Varid{suc}\;(\Varid{suc}\;\Varid{x})\;\Varid{y}\;\Varid{rewrite}\mathbin{+}\Varid{suc}\;\Varid{x}\;\Varid{y}\mathrel{=}\Varid{refl}{}\<[E]%
\ColumnHook
\end{hscode}\resethooks

  Finally, the main proof:

\begin{hscode}\SaveRestoreHook
\column{B}{@{}>{\hspre}l<{\hspost}@{}}%
\column{3}{@{}>{\hspre}l<{\hspost}@{}}%
\column{27}{@{}>{\hspre}l<{\hspost}@{}}%
\column{E}{@{}>{\hspre}l<{\hspost}@{}}%
\>[3]{}\mathbin{+}\Varid{comm}\mathbin{:}\mathbin{∀}(\Varid{x}\;\Varid{y}\mathbin{:}\Conid{ℕ})\mathbin{→}(\Varid{x}\mathbin{+}\Varid{y})\mathbin{≡}(\Varid{y}\mathbin{+}\Varid{x}){}\<[E]%
\\
\>[3]{}\mathbin{+}\Varid{comm}\;\Varid{zero}\;\Varid{y}\;\Varid{rewrite}\mathbin{+}\mathrm{0}\;\Varid{y}\mathrel{=}\Varid{refl}{}\<[E]%
\\
\>[3]{}\mathbin{+}\Varid{comm}\;(\Varid{suc}\;\Varid{x})\;\Varid{y}\;\Varid{rewrite}\mathbin{+}\Varid{suc}\;\Varid{x}\;\Varid{y}\mid {}\<[E]%
\\
\>[3]{}\hsindent{24}{}\<[27]%
\>[27]{}\mathbin{+}\Varid{suc}\;\Varid{y}\;\Varid{x}\mid {}\<[E]%
\\
\>[3]{}\hsindent{24}{}\<[27]%
\>[27]{}\mathbin{+}\Varid{comm}\;\Varid{x}\;\Varid{y}\mathrel{=}\Varid{refl}{}\<[E]%
\ColumnHook
\end{hscode}\resethooks

\begin{hscode}\SaveRestoreHook
\column{B}{@{}>{\hspre}l<{\hspost}@{}}%
\column{3}{@{}>{\hspre}l<{\hspost}@{}}%
\column{E}{@{}>{\hspre}l<{\hspost}@{}}%
\>[3]{}\Varid{open}\mathbin{≡-}\Conid{Reasoning}{}\<[E]%
\ColumnHook
\end{hscode}\resethooks

  This is the module where all the equivalence reasoning primitives are declared.
  Now, the proof which uses this module.

\begin{hscode}\SaveRestoreHook
\column{B}{@{}>{\hspre}l<{\hspost}@{}}%
\column{3}{@{}>{\hspre}l<{\hspost}@{}}%
\column{5}{@{}>{\hspre}l<{\hspost}@{}}%
\column{7}{@{}>{\hspre}l<{\hspost}@{}}%
\column{9}{@{}>{\hspre}l<{\hspost}@{}}%
\column{E}{@{}>{\hspre}l<{\hspost}@{}}%
\>[3]{}\mathbin{+}\Varid{comm2}\mathbin{:}\mathbin{∀}(\Varid{x}\;\Varid{y}\mathbin{:}\Conid{ℕ})\mathbin{→}(\Varid{x}\mathbin{+}\Varid{y})\mathbin{≡}(\Varid{y}\mathbin{+}\Varid{x}){}\<[E]%
\\
\>[3]{}\mathbin{+}\Varid{comm2}\;\Varid{zero}\;\Varid{y}\mathrel{=}{}\<[E]%
\\
\>[3]{}\hsindent{4}{}\<[7]%
\>[7]{}\Varid{begin}\;{}\<[E]%
\\
\>[7]{}\hsindent{2}{}\<[9]%
\>[9]{}\Varid{zero}\mathbin{+}\Varid{y}{}\<[E]%
\\
\>[3]{}\hsindent{4}{}\<[7]%
\>[7]{}\mathbin{≡⟨}\Varid{sym}\;(\mathbin{+}\mathrm{0}\;\Varid{y})\mathbin{⟩}{}\<[E]%
\\
\>[7]{}\hsindent{2}{}\<[9]%
\>[9]{}\Varid{y}\mathbin{+}\Varid{zero}{}\<[E]%
\\
\>[3]{}\hsindent{4}{}\<[7]%
\>[7]{}\mathbin{∎}{}\<[E]%
\\
\>[3]{}\mathbin{+}\Varid{comm2}\;(\Varid{suc}\;\Varid{x})\;\Varid{y}\mathrel{=}{}\<[E]%
\\
\>[3]{}\hsindent{2}{}\<[5]%
\>[5]{}\Varid{begin}\;{}\<[E]%
\\
\>[5]{}\hsindent{2}{}\<[7]%
\>[7]{}\Varid{suc}\;(\Varid{x}\mathbin{+}\Varid{y}){}\<[E]%
\\
\>[3]{}\hsindent{2}{}\<[5]%
\>[5]{}\mathbin{≡⟨}\Varid{cong}\;\Varid{suc}\;(\mathbin{+}\Varid{comm2}\;\Varid{x}\;\Varid{y})\mathbin{⟩}{}\<[E]%
\\
\>[5]{}\hsindent{2}{}\<[7]%
\>[7]{}\Varid{suc}\;(\Varid{y}\mathbin{+}\Varid{x}){}\<[E]%
\\
\>[3]{}\hsindent{2}{}\<[5]%
\>[5]{}\mathbin{≡⟨}\Varid{sym}\;(\mathbin{+}\Varid{suc}\;\Varid{y}\;\Varid{x})\mathbin{⟩}{}\<[E]%
\\
\>[5]{}\hsindent{2}{}\<[7]%
\>[7]{}\Varid{y}\mathbin{+}\Varid{suc}\;\Varid{x}{}\<[E]%
\\
\>[3]{}\hsindent{2}{}\<[5]%
\>[5]{}\mathbin{∎}{}\<[E]%
\ColumnHook
\end{hscode}\resethooks

  Although it is longer, due to the extra syntax, it allows reading off intermediate results, so that
  proofs become more readable. \\
  It can also aid the proving process. There are cases when you know it's
  possible to prove an equivalence from A to B inside of a bigger proof, but you want to leave that out
  for the moment and come back to it later.
  In this case, you can just introduce a hole in the ≡⟨ {!   !} ⟩ operator, and continue the main proof,
  filling in side lemmas at the end.
